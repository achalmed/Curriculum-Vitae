% Section experiencia corta
\sectionTitle{Experiencias Profesionales}{\faSuitcase}
%\renewcommand{\labelitemi}{$\bullet$}

\begin{experiences}

  \experience
  {Abril 2022}    {Docente}{Universidad Nacional Autónoma de Huanta}{Huanta}
  {Julio 2022}   {
    He tenido la oportunidad de ejercer como docente en la UNH, donde enseñé el curso de Gestión de Proyectos a estudiantes de pregrado. Esta experiencia me permitió desarrollar habilidades de enseñanza y comunicación efectiva, lo que considero una ventaja en mi perfil profesional.
    \begin{itemize}
      \item Implementación exitosa de metodologías prácticas
      \item Fomento de la participación activa
      \item Mejora de los resultados académicos
      \item Retroalimentación constructiva y personalizada
    \end{itemize}
  }
  {\footnotesize{\emph{Tecnologías utilizadas:} Trello, Google Docs, Recursos multimedia interactivos con LaTex}}
  \emptySeparator

  \experience
  {Agosto 2022}    {Practicante}{Dirección de Información Agraria y Estudios Económicos}{DRA-Ayacucho}
  {Abril 2022}   {
    Realicé mis prácticas pre profesionales en la Dirección de Información Agraria y Estudio Económicos. Durante mi tiempo en esta posición, tuve la oportunidad de contribuir en la formulación de estudios económicos y en la generación y procesamiento de información relevante para el sector agrario de la región. Estas prácticas me permitieron adquirir habilidades en análisis de datos, investigación y trabajo en equipo en un entorno profesional.
    \begin{itemize}
      \item Contribución en estudios económicos
      \item Generación y procesamiento de información relevante
      \item Análisis de datos
      \item Trabajo en equipo y colaboración
    \end{itemize}
  }
  {\footnotesize{\emph{Tecnologías utilizadas:} Excel, Stata, Power BI, R, MySQL}}
  \emptySeparator

  \experience
  {Agosto 2021}    {Secretario}{Centro de Estudiantes de Economía}{Ayacucho}
  {Julio 2020}   {
    Durante mi tiempo como estudiante de Economía, tuve la oportunidad de desempeñarme como Secretario de Prensa en el Centro de Estudiantes de dicha carrera. En este cargo, fui responsable de la difusión de las actividades y eventos del centro, así como de mantener una comunicación fluida y efectiva con los estudiantes y demás miembros del centro. Esta experiencia me permitió desarrollar habilidades en comunicación, trabajo en equipo y organización, las cuales considero importantes en mi perfil profesional.
    \begin{itemize}
      \item Gestión efectiva de la comunicación
      \item Promoción de la participación estudiantil
      \item Trabajo en equipo colaborativo
      \item Organización de eventos exitosos
    \end{itemize}
  }
  {\footnotesize{\emph{Tecnologías utilizadas:} Trello, OBS, Plataformas de redes sociales, Plataformas de gestión de contenido}}
  \emptySeparator

  \experience
  {Abril 2019} {Secretario de archivos}{Municipalidad Provincial de Cangallo.}{Cangallo}
  {Enero 2019}    {
    \begin{itemize}
      \item Oficina de archivos
    \end{itemize}
  }
  {MPC}
  \emptySeparator

  \experience
  {Diciembre 2018}  {Parlamentario Joven}{Congreso de la República}{Ayacucho}
  {Enero 2018}   {
    Como Parlamentario Joven auspiciado por la Fundación Hanns Seidel y la Oficina de Participación Ciudadana de Congeso de la Reública del Perú, he fortalecido mis capacidades de liderazgo y aprendido sobre el funcionamiento del Congreso de la República, sus comisiones de trabajo y a valorar el trabajo que se realiza, dentro de un espacio de participación y fomento de valores democráticos.
    \begin{itemize}
      \item Participación ciudadana
    \end{itemize}
  }
  {PJ2018}

\end{experiences}




